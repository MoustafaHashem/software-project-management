% ============================================================================
% CHAPTER 9: PERT TIME ESTIMATION
% ============================================================================

\chapter{PERT Time Estimation}

\vspace{12pt}

\noindent This chapter presents the Program Evaluation and Review Technique (PERT) time estimates for project activities. PERT uses three-point estimation to account for uncertainty in activity durations and provides a more realistic project completion estimate.

% ============================================================================
\section{What is PERT?}

\par \textbf{PERT} is a network analysis technique used to estimate project duration when there is a high degree of uncertainty about the individual activity duration estimates.

\par PERT uses \textbf{probabilistic time estimates} based on three scenarios for each activity:

\begin{itemize}[leftmargin=*]
    \item \textbf{Optimistic Time (O):} Best-case scenario -- minimum time if everything goes perfectly
    \item \textbf{Most Likely Time (M):} Most realistic time estimate -- what typically happens
    \item \textbf{Pessimistic Time (P):} Worst-case scenario -- maximum time if significant problems occur
\end{itemize}

\par This three-point estimate approach provides a weighted average that accounts for both best and worst-case scenarios.

% ============================================================================
\section{PERT Formula}

\par The PERT weighted average is calculated as:

\begin{equation}
\text{PERT Weighted Average} = \frac{\text{Optimistic} + 4 \times \text{Most Likely} + \text{Pessimistic}}{6}
\end{equation}

\subsection{Formula Explanation}

\par The formula gives more weight to the most likely time (4 times) while still considering both optimistic and pessimistic scenarios. This provides a more realistic estimate than simply using the most likely time alone.



\section{PERT Estimates for Project Activities}

\par The following table shows three-point time estimates for all major project activities and their PERT weighted averages.

\begin{landscape}

\begin{table}[H]
\centering
\caption{PERT Three-Point Time Estimates for Project Activities}
\small
\begin{tabularx}{\linewidth}{|c|X|c|c|c|c|l|}
\hline
\textbf{ID} & \textbf{Activity Name} & \textbf{Optimistic} & \textbf{Most Likely} & \textbf{Pessimistic} & \textbf{PERT Avg} & \textbf{Calculation} \\
 & & \textbf{(weeks)} & \textbf{(weeks)} & \textbf{(weeks)} & \textbf{(weeks)} & \\
\hline
A & Project Initiation & 2.5 & 3 & 4 & 3.1 & (2.5+4×3+4)/6 \\
\hline
B & Requirements Gathering & 1.5 & 2 & 2.5 & 2.0 & (1.5+4×2+2.5)/6 \\
\hline
C & System Design & 0.75 & 1 & 1.5 & 1.0 & (0.75+4×1+1.5)/6 \\
\hline
D & Backend Development & 4.5 & 5 & 6.5 & 5.2 & (4.5+4×5+6.5)/6 \\
\hline
E & Frontend Development & 4.5 & 5 & 6 & 5.1 & (4.5+4×5+6)/6 \\
\hline
F & System Integration & 2.5 & 3 & 4 & 3.1 & (2.5+4×3+4)/6 \\
\hline
G & System Testing & 3.5 & 4 & 5 & 4.1 & (3.5+4×4+5)/6 \\
\hline
H & User Acceptance Testing & 2.5 & 3 & 4 & 3.1 & (2.5+4×3+4)/6 \\
\hline
I & Deployment Preparation & 1.5 & 2 & 2.5 & 2.0 & (1.5+4×2+2.5)/6 \\
\hline
J & System Deployment & 1.5 & 2 & 3 & 2.1 & (1.5+4×2+3)/6 \\
\hline
K & Training \& Documentation & 3.5 & 4 & 5 & 4.1 & (3.5+4×4+5)/6 \\
\hline
L & Post-Deployment Support & 0.75 & 1 & 1.5 & 1.0 & (0.75+4×1+1.5)/6 \\
\hline
M & Project Closure & 1.5 & 2 & 2.5 & 2.0 & (1.5+4×2+2.5)/6 \\
\hline
\multicolumn{5}{|r|}{\textbf{Total PERT Duration:}} & \textbf{37.9 weeks} & \\
\hline
\end{tabularx}
\end{table}

\end{landscape}

% ============================================================================
\section{Comparison: Deterministic vs. PERT Estimates}

\par Comparing the original deterministic (single-point) estimates from the AON analysis with the PERT weighted averages:

\begin{table}[H]
\centering
\caption{Deterministic vs. PERT Duration Comparison}
\begin{tabular}{|l|c|c|c|}
\hline
\textbf{Estimate Type} & \textbf{Duration} & \textbf{Basis} & \textbf{Risk Coverage} \\
\hline
Deterministic (AON) & 28 weeks & Single estimate & Low \\
\hline
PERT Weighted Average & 37.9 weeks & Three-point estimate & High \\
\hline
\textbf{Difference} & \textbf{+9.9 weeks} & \textbf{Risk buffer} & \\
\hline
\end{tabular}
\end{table}


\section{PERT Analysis Interpretation}



\subsection{Activities with High Uncertainty}

\par Activities where pessimistic time is higher than optimistic (showing uncertainty):

\begin{itemize}[leftmargin=*]
    \item \textbf{Backend Development (D):} 4.5-6.5 weeks range (moderate risk: complex prerequisite validation)
    \item \textbf{Frontend Development (E):} 4.5-6 weeks range (moderate risk: UI/UX iterations)
    \item \textbf{System Testing (G):} 3.5-5 weeks range (moderate risk: bug discovery and fixes)
    \item \textbf{User Acceptance Testing (H):} 2.5-4 weeks range (moderate risk: stakeholder availability)
\end{itemize}

% ============================================================================
\section{Recommendations}

\par Based on the PERT analysis, the following recommendations are made:

\begin{enumerate}[leftmargin=*]
    
    \item \textbf{Monitor High-Risk Activities:} Closely track Backend Development, Frontend Development, and Testing phases
    
    \item \textbf{Early Risk Mitigation:} Start risk mitigation activities early for activities with wide time ranges
    
    \item \textbf{Regular Re-estimation:} Update PERT estimates as work progresses and uncertainty decreases
    
    \item \textbf{Communicate Realistically:} Inform stakeholders that 28 weeks is optimistic; 39=8 weeks is more realistic
\par \textbf{Final Recommendation:} Communicate a \textbf{38-40 week timeline} to stakeholders while internally targeting 28-30 weeks to maximize efficiency.
\end{enumerate}
