% ============================================================================
% CHAPTER 8: ACTIVITY ON NODE (AON) AND CRITICAL PATH
% ============================================================================

\chapter[AON Network]{Activity on Node (AON) Network and Critical Path Analysis}

\vspace{12pt}

\noindent This chapter presents the Activity-on-Node (AON) network diagram for the Online Course Registration Portal project. The AON technique is used to visualize project workflow, identify activity dependencies, and determine the critical path that defines the minimum project duration.

% ============================================================================
\section{Introduction to AON and Critical Path Method}

\par The Activity-on-Node (AON) network diagram represents project activities as nodes (boxes) and dependencies as arrows connecting them. This approach, also known as Precedence Diagramming Method (PDM), provides a clear visualization of:

\begin{itemize}[leftmargin=*]
    \item The sequence of project activities
    \item Dependencies and relationships between activities
    \item The critical path (longest path determining project duration)
    \item Activities with scheduling flexibility (slack)
\end{itemize}

\subsection{Critical Path Method (CPM)}

\par The Critical Path Method uses forward and backward pass calculations to determine:

\begin{itemize}[leftmargin=*]
    \item \textbf{ES (Early Start):} Earliest time an activity can begin
    \item \textbf{EF (Early Finish):} Earliest time an activity can complete (ES + Duration)
    \item \textbf{LS (Late Start):} Latest time an activity can begin without delaying the project
    \item \textbf{LF (Late Finish):} Latest time an activity can complete without delaying the project
    \item \textbf{Total Slack:} Amount of time an activity can be delayed without delaying the project (LS - ES)
    \item \textbf{Free Slack:} Amount of time an activity can be delayed without delaying any successor (ES of successor - EF)
\end{itemize}

\par Activities with zero total slack form the \textbf{critical path} -- any delay in these activities delays the entire project.

% ============================================================================
\section{High-Level Activity List with CPM Calculations (ES, EF, LS, LF, Free Slack, Total Slack)}



\begin{landscape}

\begin{table}[H]
\centering
\caption{High-Level Activity List with CPM Calculations}
\footnotesize
\begin{tabularx}{\linewidth}{|c|X|c|c|c|c|c|c|c|c|}
\hline
\textbf{ID} & \textbf{Activity Name (WBS)} & \textbf{Duration} & \textbf{Pred.} & \textbf{ES} & \textbf{EF} & \textbf{LS} & \textbf{LF} & \textbf{Free Slack} & \textbf{Total Slack} \\
\hline
A & Project Initiation (1.1) & 3 weeks & - & 0 & 3 & 0 & 3 & 0 & 0 \\
\hline
B & Requirements Gathering (1.2) & 2 weeks & A & 3 & 5 & 3 & 5 & 0 & 0 \\
\hline
C & System Design (1.3) & 1 week & B & 5 & 6 & 5 & 6 & 0 & 0 \\
\hline
D & Backend Development (1.4.1) & 5 weeks & C & 6 & 11 & 6 & 11 & 0 & 0 \\
\hline
E & Frontend Development (1.4.2) & 5 weeks & C & 6 & 11 & 6 & 11 & 0 & 0 \\
\hline
F & System Integration (1.4.3) & 3 weeks & D, E & 11 & 14 & 11 & 14 & 0 & 0 \\
\hline
G & System Testing (1.5.2) & 4 weeks & F & 14 & 18 & 14 & 18 & 0 & 0 \\
\hline
H & User Acceptance Testing (1.5.3) & 3 weeks & G & 18 & 21 & 18 & 21 & 0 & 0 \\
\hline
I & Deployment Preparation (1.6.1) & 2 weeks & H & 21 & 23 & 21 & 23 & 0 & 0 \\
\hline
J & System Deployment (1.6.2--1.6.3) & 2 weeks & I & 23 & 25 & 23 & 25 & 0 & 0 \\
\hline
K & Training \& Documentation (1.7) & 4 weeks & F & 14 & 18 & 22 & 26 & 8 weeks & 8 weeks \\
\hline
L & Post-Deployment Support (1.6.4) & 1 week & J & 25 & 26 & 25 & 26 & 0 & 0 \\
\hline
M & Project Closure (1.8) & 2 weeks & L, K & 26 & 28 & 26 & 28 & 0 & 0 \\
\hline
\end{tabularx}
\end{table}

\end{landscape}

% ============================================================================
\begin{figure}[H]
    \centering
    \includegraphics[width=\linewidth]{images/AON.png} 
    \caption{Activity on Node (AON) Network Diagram - critical path activities surrounded by red boxes}
    \label{fig:aon-network}
\end{figure}
\section{Forward Pass Calculation (ES and EF)}

\par The forward pass determines the earliest times activities can start and finish, working from project start to finish.

\subsection{Calculation Method}

\begin{itemize}[leftmargin=*]
    \item \textbf{ES = } Maximum EF of all predecessor activities (or 0 for start activity)
    \item \textbf{EF = } ES + Duration
\end{itemize}

\par \textit{Note: Activities with ES = LS and EF = LF have zero total slack and are on the critical path.}

\subsection{Forward Pass Results}

\begin{enumerate}[leftmargin=*]
    \item \textbf{Activity A (Project Initiation):} ES = 0, EF = 0 + 3 = 3
    \item \textbf{Activity B (Requirements):} ES = 3, EF = 3 + 2 = 5
    \item \textbf{Activity C (Design):} ES = 5, EF = 5 + 1 = 6
    \item \textbf{Activity D (Backend):} ES = 6, EF = 6 + 5 = 11
    \item \textbf{Activity E (Frontend):} ES = 6, EF = 6 + 5 = 11 (must finish on time for F)
    \item \textbf{Activity F (Integration):} ES = max(11, 11) = 11, EF = 11 + 3 = 14
    \item \textbf{Activity G (System Testing):} ES = 14, EF = 14 + 4 = 18
    \item \textbf{Activity H (UAT):} ES = 18, EF = 18 + 3 = 21
    \item \textbf{Activity I (Deployment Prep):} ES = 21, EF = 21 + 2 = 23
    \item \textbf{Activity J (Deployment):} ES = 23, EF = 23 + 2 = 25
    \item \textbf{Activity K (Training):} ES = 14, EF = 14 + 4 = 18
    \item \textbf{Activity L (Support):} ES = 25, EF = 25 + 1 = 26
    \item \textbf{Activity M (Closure):} ES = max(26, 18) = 26, EF = 26 + 2 = 28
\end{enumerate}

\par \textbf{Project Duration = 28 weeks} (from forward pass calculation)

% ============================================================================
\section{Backward Pass Calculation (LS and LF)}

\par The backward pass determines the latest times activities can start and finish without delaying the project, working from project finish to start.

\subsection{Calculation Method}

\begin{itemize}[leftmargin=*]
    \item \textbf{LF = } Minimum LS of all successor activities (or project duration for final activity)
    \item \textbf{LS = } LF - Duration
\end{itemize}

\subsection{Backward Pass Results}

\begin{enumerate}[leftmargin=*]
    \item \textbf{Activity M (Closure):} LF = 28, LS = 28 - 2 = 26
    \item \textbf{Activity L (Support):} LF = 26, LS = 26 - 1 = 25
    \item \textbf{Activity K (Training):} LF = 26, LS = 26 - 4 = 22 
    \item \textbf{Activity J (Deployment):} LF = 25, LS = 25 - 2 = 23
    \item \textbf{Activity I (Deployment Prep):} LF = 23, LS = 23 - 2 = 21
    \item \textbf{Activity H (UAT):} LF = 21, LS = 21 - 3 = 18
    \item \textbf{Activity G (System Testing):} LF = 18, LS = 18 - 4 = 14
    \item \textbf{Activity F (Integration):} LF = min(14, 22) = 14, LS = 14 - 3 = 11
    \item \textbf{Activity E (Frontend):} LF = 11, LS = 11 - 5 = 6 (critical - required for F)
    \item \textbf{Activity D (Backend):} LF = 11, LS = 11 - 5 = 6
    \item \textbf{Activity C (Design):} LF = min(6, 7) = 6, LS = 6 - 1 = 5
    \item \textbf{Activity B (Requirements):} LF = 5, LS = 5 - 2 = 3
    \item \textbf{Activity A (Initiation):} LF = 3, LS = 3 - 3 = 0
\end{enumerate}

% ============================================================================
\section{Slack Analysis}

\par Slack (or float) represents the amount of time an activity can be delayed. There are two types of slack:

\subsection{Types of Slack}

\begin{itemize}[leftmargin=*]
    \item \textbf{Total Slack:} Maximum time an activity can be delayed without delaying project completion
    \item \textbf{Free Slack:} Maximum time an activity can be delayed without delaying the early start of any successor activity
\end{itemize}

\subsection{Slack Calculations}

\par \textbf{Total Slack = LS - ES} (or equivalently, LF - EF)

\par \textbf{Free Slack = ES(successor) - EF(current)} (minimum if multiple successors)

\subsection{Slack Calculations Explained}

\begin{itemize}[leftmargin=*]
    \item \textbf{Activity K (Training):}
    \begin{itemize}
        \item Total Slack = LS - ES = 22 - 14 = 8 weeks
        \item Free Slack = ES(M) - EF(K) = 26 - 18 = 8 weeks
        \item Can be delayed 8 weeks without affecting project or successor M
    \end{itemize}
    \item \textbf{All other activities:}
    \begin{itemize}
        \item Both Total Slack and Free Slack = 0
        \item Form the critical path
        \item Any delay immediately impacts project schedule
    \end{itemize}
\end{itemize}

\subsection{Slack Implications}

\begin{itemize}[leftmargin=*]
    \item \textbf{Zero Slack Activities:} Must start and finish on schedule; form the critical path
    \item \textbf{Frontend Development (E):} Has 0 total and free slack (critical); must finish by week 11 as System Integration depends on it
    \item \textbf{Training (K):} Has 8 weeks of both total and free slack; can start anytime between weeks 14-22 without delaying project or successor
    \item \textbf{Equal Total and Free Slack:} When total slack equals free slack (as in Activity K), the activity can use all its slack without impacting any other activity
\end{itemize}

% ============================================================================
\section{Critical Path Identification}

\par The critical path is the sequence of activities with zero float, representing the longest path through the project network.

\subsection{Critical Path}

\begin{center}
\Large
\textbf{A → B → C → \{D, E\} → F → G → H → I → J → L → M}
\end{center}

\par \textit{Note: Activities D (Backend) and E (Frontend) are parallel critical activities that must both complete before F (Integration) can begin. The notation \{D, E\} indicates these activities run concurrently.}

\begin{table}[H]
\centering
\caption{Critical Path Activities}
\begin{tabular}{|c|l|c|c|}
\hline
\textbf{Activity} & \textbf{Activity Name} & \textbf{Duration} & \textbf{Dates (2026)} \\
\hline
A & Project Initiation & 3 weeks & Mar 1 -- Mar 21 \\
\hline
B & Requirements Gathering & 2 weeks & Mar 22 -- Apr 4 \\
\hline
C & System Design & 1 week & Apr 5 -- Apr 11 \\
\hline
D & Backend Development & 5 weeks & Apr 12 -- May 16 \\
\hline
E & Frontend Development & 5 weeks & Apr 12 -- May 16 \\
\hline
F & System Integration & 3 weeks & May 17 -- Jun 6 \\
\hline
G & System Testing & 4 weeks & Jun 7 -- Jul 4 \\
\hline
H & User Acceptance Testing & 3 weeks & Jul 5 -- Jul 25 \\
\hline
I & Deployment Preparation & 2 weeks & Jul 26 -- Aug 8 \\
\hline
J & System Deployment & 2 weeks & Aug 9 -- Aug 22 \\
\hline
L & Post-Deployment Support & 1 week & Aug 23 -- Aug 29 \\
\hline
M & Project Closure & 2 weeks & Aug 30 -- Sep 12 \\
\hline
\multicolumn{3}{|r|}{\textbf{Total Project Duration:}} & \textbf{28 weeks} \\
\hline
\end{tabular}
\end{table}

\subsection{Critical Path Significance}

\begin{itemize}[leftmargin=*]
    \item Any delay in critical path activities directly delays project completion
    \item Critical activities require closest monitoring and management attention
    \item \textbf{Parallel Critical Activities:} Both Backend (D) and Frontend (E) are critical activities running concurrently (weeks 6--11). System Integration (F) cannot start until BOTH are complete, making both paths critical
    \item Resources should be prioritized for critical path activities
    
\end{itemize}


\section{Critical Path Management}

\subsection{Management Strategies for Critical Activities}

\begin{enumerate}[leftmargin=*]
    \item \textbf{Resource Priority:} Allocate best resources to critical path activities (A, B, C, D, E, F, G, H, I, J, L, M)
    \item \textbf{Close Monitoring:} Track progress daily on critical activities; weekly for non-critical
    \item \textbf{Risk Mitigation:} Develop contingency plans for high-risk critical activities
    \item \textbf{Early Warning System:} Implement alerts when critical activities fall behind schedule
    \item \textbf{Buffer Management:} Consider adding time buffers before key milestones on critical path
\end{enumerate}


\subsection{Managing Non-Critical Activities}

\begin{itemize}[leftmargin=*]
    \item \textbf{Training \& Documentation (K):} 8 weeks float provides significant scheduling flexibility; can start anytime between weeks 14-22
    \item Use float in Training activities to balance workload and optimize resource utilization
    \item Training can be scheduled flexibly without impacting the critical deployment path
\end{itemize}

% ============================================================================
\section{Network Analysis Summary}

\subsection{Key Findings}

\begin{enumerate}[leftmargin=*]
    \item \textbf{Project Duration:} 28 weeks (Mar 1 -- Sep 15, 2026)
    \item \textbf{Critical Path:} 12 out of 13 activities are critical (92\% of activities)
    \item \textbf{Critical Path Length:} 28 weeks (no schedule reserve)
    \item \textbf{Parallel Execution:} Backend (D) and Frontend (E) run in parallel but both are critical
    \item \textbf{Slack Available:} Very limited (only Training with 8 weeks total and free slack)
\end{enumerate}

\subsection{Schedule Risk Assessment}

\begin{itemize}[leftmargin=*]
    \item \textbf{High Risk:} 92\% of activities are critical with zero total slack
    \item \textbf{Limited Flexibility:} Very little schedule buffer available
 
    \item \textbf{Risk Mitigation:} Focus on preventing delays in both Backend (D) and Frontend (E), Integration (F), Testing (G, H), and Deployment (I, J)
\end{itemize}

\subsection{Recommendations}

\begin{enumerate}[leftmargin=*]
    \item Assign most experienced resources to critical path activities
    \item Implement daily progress tracking for activities D, E, F, G, H, I, J
    \item Use 8 weeks of slack in Training (K) to balance resource allocation
\end{enumerate}
