% ============================================================================
% CHAPTER 14: RISK ANALYSIS
% ============================================================================

\chapter{Risk Analysis}

\vspace{12pt}

\noindent This chapter presents Project Risk Management for the Online Course Registration Portal using the six core processes: Plan Risk Management, Identify Risks, Perform Qualitative Risk Analysis, Perform Quantitative Risk Analysis, Plan Risk Responses, and Monitor and Control Risks.

% ============================================================================
\section{Plan Risk Management}

\par This process defines how risk management will be performed for the project.

\subsection{Risk Management Plan}

\begin{table}[H]
\centering
\caption{Risk Management Plan Elements}
\begin{tabularx}{\textwidth}{|l|X|}
\hline
\textbf{Element} & \textbf{Description} \\
\hline
Methodology & Qualitative analysis for all risks; quantitative analysis for high-impact risks \\
\hline
Roles \& Responsibilities & Project Manager owns risk register; Team Leads own technical risks \\
\hline
Budget & 145,000 EGP contingency reserve (10\% of 1,450,000 base) \\
\hline
Schedule & 2 weeks buffer included in 36-week timeline \\
\hline
Risk Categories & Technical, Organizational, Project Management, External \\
\hline
Probability \& Impact & Low/Medium/High scales with defined thresholds \\
\hline
Documentation & Risk Register updated weekly; reported in status meetings \\
\hline
\end{tabularx}
\end{table}

% ============================================================================
\section{Identify Risks}

\par This process determines which risks might affect the project and documents their characteristics.

\subsection{Risk Identification Tools}

\begin{table}[H]
\centering
\caption{Risk Identification Methods Used}
\begin{tabular}{|l|l|}
\hline
\textbf{Tool} & \textbf{Purpose} \\
\hline
Brainstorming & Generate a broad list of possible risks \\
\hline
Interviews & Learn from experienced stakeholders \\
\hline
SWOT Analysis & Identify strengths, weaknesses, opportunities, threats \\
\hline
\end{tabular}
\end{table}

\subsection{Risk Register}

\begin{landscape}

\begin{table}[H]
\centering
\caption{Project Risk Register}
\scriptsize
\begin{tabularx}{\linewidth}{|c|X|l|X|X|c|c|l|l|l|}
\hline
\textbf{ID} & \textbf{Risk Description} & \textbf{Category} & \textbf{Root Cause} & \textbf{Trigger} & \textbf{Prob.} & \textbf{Impact} & \textbf{Owner} & \textbf{Response} & \textbf{Status} \\
\hline
R1 & SIS integration more complex than anticipated & Technical & Legacy system documentation gaps & Integration tests fail & H & H & SA & Mitigate & Open \\
\hline
R2 & System performance degrades under concurrent load & Technical & Insufficient load testing & Response time > 3s & M & H & TL & Mitigate & Open \\
\hline
R3 & Prerequisite validation logic contains errors & Technical & Complex business rules & Failed UAT tests & M & H & Dev & Mitigate & Open \\
\hline
R4 & Key developer leaves mid-project & Organizational & Market competition & Resignation notice & M & H & PM & Transfer & Open \\
\hline
R5 & Requirements gathering delayed & PM & Stakeholder unavailability & Missed meetings & H & M & BA & Mitigate & Open \\
\hline
<<<<<<< HEAD
R6 & Budget overrun exceeds 145,000 EGP reserve & PM & Scope creep & Cost variance $>$ 10\% & M & M & PM & Avoid & Open \\
\hline
R7 & Schedule slips beyond 36 weeks & PM & Underestimated complexity & SPI $<$ 0.9 & M & H & PM & Mitigate & Open \\
=======
R6 & Budget overrun exceeds 145,000 EGP reserve & PM & Scope creep & Cost variance > 10\% & M & M & PM & Avoid & Open \\
\hline
R7 & Schedule slips beyond 36 weeks & PM & Underestimated complexity & SPI < 0.9 & M & H & PM & Mitigate & Open \\
>>>>>>> 4cb30038012f55466cd5cc26b4c5c001b2cac309
\hline
R8 & Staff resistance to new system & Organizational & Change aversion & Training attendance low & H & M & PM & Mitigate & Open \\
\hline
R9 & Security vulnerabilities discovered & Technical & Inadequate security review & Penetration test fails & L & H & SecEng & Mitigate & Open \\
\hline
R10 & Vendor discontinues critical component & External & Market changes & EOL announcement & L & H & SA & Transfer & Open \\
\hline
\end{tabularx}
\end{table}

\end{landscape}

% ============================================================================
\section{Perform Qualitative Risk Analysis}

\par This process prioritizes risks based on probability of occurrence and impact on project objectives. The output is a ranked list of risks, helping the team focus on the most significant ones.

\subsection{Probability and Impact Scales}

\begin{table}[H]
\centering
\caption{Probability and Impact Definitions}
\begin{tabular}{|l|c|l|l|}
\hline
\textbf{Level} & \textbf{Probability} & \textbf{Impact on Schedule} & \textbf{Impact on Budget} \\
\hline
<<<<<<< HEAD
Low (L) & $<$ 30\% & $<$ 1 week delay & $<$ 50,000 EGP \\
\hline
Medium (M) & 30-60\% & 1-3 weeks delay & 50,000-150,000 EGP \\
\hline
High (H) & $>$ 60\% & $>$ 3 weeks delay & $>$ 150,000 EGP \\
=======
Low (L) & < 30\% & < 1 week delay & < 50,000 EGP \\
\hline
Medium (M) & 30-60\% & 1-3 weeks delay & 50,000-150,000 EGP \\
\hline
High (H) & > 60\% & > 3 weeks delay & > 150,000 EGP \\
>>>>>>> 4cb30038012f55466cd5cc26b4c5c001b2cac309
\hline
\end{tabular}
\end{table}

\subsection{Probability-Impact Matrix}

\begin{table}[H]
\centering
\caption{Risk Probability-Impact Matrix}
\begin{tabular}{|l|c|c|c|}
\hline
\textbf{Probability $\backslash$ Impact} & \textbf{Low} & \textbf{Medium} & \textbf{High} \\
\hline
<<<<<<< HEAD
\textbf{High} & Medium Risk & High Risk (R5, R8) & \textbf{Critical Risk (R1)} \\
=======
\textbf{High} & Medium Risk & High Risk (R5, R8) & \cellcolor{red!30}\textbf{Critical Risk (R1)} \\
>>>>>>> 4cb30038012f55466cd5cc26b4c5c001b2cac309
\hline
\textbf{Medium} & Low Risk & Medium Risk (R6) & High Risk (R2, R3, R4, R7) \\
\hline
\textbf{Low} & Low Risk & Low Risk & Medium Risk (R9, R10) \\
\hline
\end{tabular}
\end{table}

% ============================================================================
\section{Perform Quantitative Risk Analysis}

\par This process numerically estimates the effect of risks on project objectives. It is mainly used for large or complex projects.

\subsection{Sensitivity Analysis}

\par Sensitivity analysis examines how changes in key variables affect project outcomes:

\begin{table}[H]
\centering
\caption{Sensitivity Analysis Results}
\begin{tabular}{|l|l|l|}
\hline
\textbf{Variable Changed} & \textbf{Outcome Affected} & \textbf{Result} \\
\hline
Development cost +20\% & Total budget & Budget increases by 160,000 EGP \\
\hline
Task duration +15\% & Project schedule & Delay risk of 5 weeks \\
\hline
<<<<<<< HEAD
Resource count $-$1 developer & Productivity & Output decreases, schedule extends \\
=======
Resource count -1 developer & Productivity & Output decreases, schedule extends \\
>>>>>>> 4cb30038012f55466cd5cc26b4c5c001b2cac309
\hline
Integration effort +50\% & Schedule and budget & Significant delay and cost overrun \\
\hline
\end{tabular}
\end{table}

% ============================================================================
\section{Plan Risk Responses}

\par This process defines actions to address identified risks.

\subsection{Response Strategies for Negative Risks (Threats)}

\begin{table}[H]
\centering
\caption{Threat Response Strategies}
\begin{tabular}{|l|l|}
\hline
\textbf{Strategy} & \textbf{Description} \\
\hline
Acceptance & Acknowledge the risk without action \\
\hline
Avoidance & Change the plan to eliminate the risk \\
\hline
Mitigation & Reduce probability or impact \\
\hline
Transference & Shift risk to a third party \\
\hline
\end{tabular}
\end{table}

\subsection{Response Strategies for Positive Risks (Opportunities)}

\begin{table}[H]
\centering
\caption{Opportunity Response Strategies}
\begin{tabular}{|l|l|}
\hline
\textbf{Strategy} & \textbf{Description} \\
\hline
Acceptance & Take advantage if it occurs \\
\hline
Exploitation & Ensure the opportunity happens \\
\hline
Enhancement & Increase probability or impact \\
\hline
Sharing & Partner with others to realize the opportunity \\
\hline
\end{tabular}
\end{table}

\subsection{Risk Response Plan}

\begin{table}[H]
\centering
\caption{Risk Response Plan}
\small
\begin{tabularx}{\textwidth}{|c|l|X|X|}
\hline
\textbf{ID} & \textbf{Strategy} & \textbf{Response Action} & \textbf{Contingency Plan} \\
\hline
R1 & Mitigate & Early prototype; dedicated integration specialist & Manual data procedures temporarily \\
\hline
R2 & Mitigate & Load testing in Phase 3; scalable architecture & Add cloud resources on-demand \\
\hline
R3 & Mitigate & Automated unit tests; peer code review & Extended UAT cycle \\
\hline
R4 & Transfer & Cross-training; documentation & Contract backup developers \\
\hline
R5 & Mitigate & Schedule interviews early; async reviews & Use documented assumptions \\
\hline
R6 & Avoid & Strict change control; scope freeze & Use contingency reserve \\
\hline
R7 & Mitigate & Weekly schedule tracking; buffer time & Fast-track critical path \\
\hline
R8 & Mitigate & Change management program; early training & Executive communication \\
\hline
R9 & Mitigate & Security review in design; testing & Emergency patch process \\
\hline
R10 & Transfer & Contractual SLAs; alternatives identified & Switch to backup vendor \\
\hline
\end{tabularx}
\end{table}

\subsection{Contingency Reserves}

\begin{table}[H]
\centering
\caption{Contingency Reserve Allocation}
\begin{tabular}{|l|r|l|}
\hline
\textbf{Reserve Type} & \textbf{Amount} & \textbf{Purpose} \\
\hline
Budget Reserve & 145,000 EGP (10\%) & Cost overrun protection \\
\hline
Schedule Reserve & 2 weeks & Buffer for schedule risks \\
\hline
\end{tabular}
\end{table}

% ============================================================================
\section{Monitor and Control Risks}

\par This is an ongoing process throughout the project lifecycle.

\subsection{Monitoring Activities}

\begin{itemize}[leftmargin=*]
    \item Track identified risks
    \item Identify new risks
    \item Execute response plans
    \item Evaluate response effectiveness
\end{itemize}


